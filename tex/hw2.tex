\documentclass{article}

%%%%%%%%%%%%%%%%%%%%%%%%%%%%%%%%%%%%%%%%%
% Lachaise Assignment
% Structure Specification File
% Version 1.0 (26/6/2018)
%
% This template originates from:
% http://www.LaTeXTemplates.com
%
% Authors:
% Marion Lachaise & François Févotte
% Vel (vel@LaTeXTemplates.com)
%
% License:
% CC BY-NC-SA 3.0 (http://creativecommons.org/licenses/by-nc-sa/3.0/)
% 
%%%%%%%%%%%%%%%%%%%%%%%%%%%%%%%%%%%%%%%%%

%----------------------------------------------------------------------------------------
%	PACKAGES AND OTHER DOCUMENT CONFIGURATIONS
%----------------------------------------------------------------------------------------

\usepackage{amsmath,amsfonts,stmaryrd,amssymb} % Math packages

\usepackage{enumerate} % Custom item numbers for enumerations

\usepackage[ruled]{algorithm2e} % Algorithms

\usepackage[framemethod=tikz]{mdframed} % Allows defining custom boxed/framed environments

\usepackage{listings} % File listings, with syntax highlighting
\lstset{
	basicstyle=\ttfamily, % Typeset listings in monospace font
}

%----------------------------------------------------------------------------------------
%	DOCUMENT MARGINS
%----------------------------------------------------------------------------------------

\usepackage{geometry} % Required for adjusting page dimensions and margins

\geometry{
	paper=a4paper, % Paper size, change to letterpaper for US letter size
	top=2.5cm, % Top margin
	bottom=3cm, % Bottom margin
	left=2.5cm, % Left margin
	right=2.5cm, % Right margin
	headheight=14pt, % Header height
	footskip=1.5cm, % Space from the bottom margin to the baseline of the footer
	headsep=1.2cm, % Space from the top margin to the baseline of the header
	%showframe, % Uncomment to show how the type block is set on the page
}

%----------------------------------------------------------------------------------------
%	FONTS
%----------------------------------------------------------------------------------------

\usepackage[utf8]{inputenc} % Required for inputting international characters
\usepackage[T1]{fontenc} % Output font encoding for international characters

\usepackage{XCharter} % Use the XCharter fonts
\usepackage{graphicx} % Required for including images
%----------------------------------------------------------------------------------------
%	COMMAND LINE ENVIRONMENT
%----------------------------------------------------------------------------------------

% Usage:
% \begin{commandline}
%	\begin{verbatim}
%		$ ls
%		
%		Applications	Desktop	...
%	\end{verbatim}
% \end{commandline}

\mdfdefinestyle{commandline}{
	leftmargin=10pt,
	rightmargin=10pt,
	innerleftmargin=15pt,
	middlelinecolor=black!50!white,
	middlelinewidth=2pt,
	frametitlerule=false,
	backgroundcolor=black!5!white,
	frametitle={Command Line},
	frametitlefont={\normalfont\sffamily\color{white}\hspace{-1em}},
	frametitlebackgroundcolor=black!50!white,
	nobreak,
}

% Define a custom environment for command-line snapshots
\newenvironment{commandline}{
	\medskip
	\begin{mdframed}[style=commandline]
}{
	\end{mdframed}
	\medskip
}

%----------------------------------------------------------------------------------------
%	FILE CONTENTS ENVIRONMENT
%----------------------------------------------------------------------------------------

% Usage:
% \begin{file}[optional filename, defaults to "File"]
%	File contents, for example, with a listings environment
% \end{file}

\mdfdefinestyle{file}{
	innertopmargin=1.6\baselineskip,
	innerbottommargin=0.8\baselineskip,
	topline=false, bottomline=false,
	leftline=false, rightline=false,
	leftmargin=2cm,
	rightmargin=2cm,
	singleextra={%
		\draw[fill=black!10!white](P)++(0,-1.2em)rectangle(P-|O);
		\node[anchor=north west]
		at(P-|O){\ttfamily\mdfilename};
		%
		\def\l{3em}
		\draw(O-|P)++(-\l,0)--++(\l,\l)--(P)--(P-|O)--(O)--cycle;
		\draw(O-|P)++(-\l,0)--++(0,\l)--++(\l,0);
	},
	nobreak,
}

% Define a custom environment for file contents
\newenvironment{file}[1][File]{ % Set the default filename to "File"
	\medskip
	\newcommand{\mdfilename}{#1}
	\begin{mdframed}[style=file]
}{
	\end{mdframed}
	\medskip
}

%----------------------------------------------------------------------------------------
%	NUMBERED QUESTIONS ENVIRONMENT
%----------------------------------------------------------------------------------------

% Usage:
% \begin{question}[optional title]
%	Question contents
% \end{question}

\mdfdefinestyle{question}{
	innertopmargin=1.2\baselineskip,
	innerbottommargin=0.8\baselineskip,
	roundcorner=5pt,
	nobreak,
	singleextra={%
		\draw(P-|O)node[xshift=1em,anchor=west,fill=white,draw,rounded corners=5pt]{%
		Question \theQuestion\questionTitle};
	},
}

\newcounter{Question} % Stores the current question number that gets iterated with each new question

% Define a custom environment for numbered questions
\newenvironment{question}[1][\unskip]{
	\bigskip
	\stepcounter{Question}
	\newcommand{\questionTitle}{~#1}
	\begin{mdframed}[style=question]
}{
	\end{mdframed}
	\medskip
}

%----------------------------------------------------------------------------------------
%	WARNING TEXT ENVIRONMENT
%----------------------------------------------------------------------------------------

% Usage:
% \begin{warn}[optional title, defaults to "Warning:"]
%	Contents
% \end{warn}

\mdfdefinestyle{warning}{
	topline=false, bottomline=false,
	leftline=false, rightline=false,
	nobreak,
	singleextra={%
		\draw(P-|O)++(-0.5em,0)node(tmp1){};
		\draw(P-|O)++(0.5em,0)node(tmp2){};
		\fill[black,rotate around={45:(P-|O)}](tmp1)rectangle(tmp2);
		\node at(P-|O){\color{white}\scriptsize\bf !};
		\draw[very thick](P-|O)++(0,-1em)--(O);%--(O-|P);
	}
}

% Define a custom environment for warning text
\newenvironment{warn}[1][Warning:]{ % Set the default warning to "Warning:"
	\medskip
	\begin{mdframed}[style=warning]
		\noindent{\textbf{#1}}
}{
	\end{mdframed}
}

%----------------------------------------------------------------------------------------
%	INFORMATION ENVIRONMENT
%----------------------------------------------------------------------------------------

% Usage:
% \begin{info}[optional title, defaults to "Info:"]
% 	contents
% 	\end{info}

\mdfdefinestyle{info}{%
	topline=false, bottomline=false,
	leftline=false, rightline=false,
	nobreak,
	singleextra={%
		\fill[black](P-|O)circle[radius=0.4em];
		\node at(P-|O){\color{white}\scriptsize\bf i};
		\draw[very thick](P-|O)++(0,-0.8em)--(O);%--(O-|P);
	}
}

% Define a custom environment for information
\newenvironment{info}[1][Info:]{ % Set the default title to "Info:"
	\medskip
	\begin{mdframed}[style=info]
		\noindent{\textbf{#1}}
}{
	\end{mdframed}
}
 % Include the file specifying the document structure and custom commands

\title{Probability and Stochastic Processes (1) }
\author{Yu Yangcheng, 2023010719\\ Tsinghua University}
\date{\textbf{Feb. 27, 2025}}

\begin{document}

\maketitle

\section*{Problem Set 2}


\begin{question}
You enter a special kind of chess tournament, in which you play one game with each of three opponents, but you get to choose the order in which you play your opponents, knowing the probability of a win against each. You win the tournament if you win two games in a row, and you want to maximize the probability of winning. Show that it is optimal to play the weakest opponent second, and that the order of playing the other two opponents does not matter.
\end{question}

    \textbf{Proof.}    
    Let the order of the game be a, b, and c, then the probability of winning the tournament is $$ P(a)P(b)+P(b)P(c)-P(a)P(b)P(c) ,$$
    where \( P(a) \) is the probability of winning the first game, and so on. Obviously, we have $P(a)P(b)P(c)=constant$, so we need to proof that \( P(a)P(b) +P(b)P(c) \) is maximized when b is the weakest opponent. Since \(P(a)P(b)+P(a)P(c)+P(b)P(c)\), that's equivalent to proof that \( P(a)P(c) \) is minimized when b is the weakest opponent. That's trivial since $P(a)$, $P(b)$ are the smallest two probabilities. Additionally, the order of playing the other two opponents doesn't change the value of $P(a)P(c)$. Therefore, the statement is true.


\begin{question}
Show that a countable intersection of events with probability 1 still has probability 1.
\end{question}

    \textbf{Proof.}
    Let's consider the sequence of events $A_1,A_2,\dots $ such that 
    $$ P(A_1)=P(A_2)=\dots=1 $$
    Consider the implement of intersections of $A_i$, $i.e.$
    $$ (\bigcap_{i=1}^{\infty} A_i)^c=\bigcup _{i=1}^{\infty}A_i^c $$
    Given the possibility of event $A_i$ is 1, we have 
    $$ P(\bigcup _{i=1}^{\infty}A_i^c)\leq \sum_{i=1}^{\infty}P(A_i^c)=\sum_{i=1}^{\infty}0=0 $$
    Therefore, 
    $$ P(\bigcap_{i=1}^{\infty} A_i)=1-P((\bigcap_{i=1}^{\infty})^c)=1 $$

\begin{question}
We are given three coins: one has heads on both faces, the second has tails on both faces, and the third has a head on one face and a tail on the other. We choose a coin at random, toss it, and the result is heads. What is the probability that the opposite face is tails?
\end{question}

    \textbf{Proof.}
    Let A = The result is head, B = The opposite face is tails. Then 
    $$ P(A)=\frac{1}{3}\times 1+\frac{1}{3}\times 0+\frac{1}{3}\times \frac{1}{2}=\frac{1}{2} $$ 
    And 
    $$ P(AB)=\frac{1}{3} $$
    Under Bayes's theorem, 
    $$ P(B|A)=\frac{P(AB)}{P(A)}=\frac{2}{3} $$
    That's the probability that the opposite face is tails.

\begin{question}
Each of \(k\) jars contains \(m\) white and \(n\) black balls. A ball is randomly chosen from jar 1 and transferred to jar 2, then a ball is randomly chosen from jar 2 and transferred to jar 3, etc. Finally, a ball is randomly chosen from jar \(k\). Show that the probability that the last ball is white is the same as the probability that the first ball is white, i.e., it is \( \frac{m}{m + n} \).
\end{question}

    \textbf{Proof.}
    Let P(the i th. ball is white)=$P_i$. We should prove that $P_i\equiv \frac{m}{m+n}$. We verify the base case, when i = 1, the statement is obviously right. Assume the statement holds for $i=p$, that is, 
    $$ P(\text{the p th. ball is white})=\frac{m}{m+n} $$
    Now, we prove that the statement holds for p+1: 
    $$ P_{p+1}=P_p\frac{m+1}{n+m+1}+(1-P_p) \frac{m}{n+m+1}= \frac{m}{m+n}\frac{m+1}{m+n+1}+\frac{n}{m+n}\frac{m}{m+n+1}=\frac{m}{m+n}$$
    The proposition is proved by mathematical induction.

\begin{question}
Suppose we would like to represent an infinite sequence of binary observations, where each observation is a zero or one with equal probability. For example, the experiment could consist of repeatedly flipping a fair coin, and recording a one each time it shows heads and a zero each time it shows tails. Then an outcome \( \omega \) would be an infinite sequence, \( \omega = (\omega_1, \omega_2, ...) \), such that for each \(i \geq 1\), \( \omega_i \in \{0, 1\} \). Let \( \Omega \) be the set of all such \( \omega \)’s. The associated \( \sigma \)-algebra \( \mathcal{F} \) satisfies that any set that can be defined in terms of finitely many of the observations is in \( \mathcal{F} \). In particular, for any binary sequence \( (b_1, b_2, ..., b_k) \) of some finite length \( k \), the set \( \{ \omega \in \Omega : w_i = b_i \text{ for } 1 \leq i \leq k \} \) should be in \( \mathcal{F} \) with a probability of \( 2^{-k} \). Suppose that there are two players who take turns performing the coin flips, with the first one to get heads wins. Let \( F \) be the event that the player going first wins. Show that \( F \in \mathcal{F} \) and find \( P(F) \).
\end{question}

    \textbf{Proof.}
    \begin{enumerate}
        \item \( F \in \mathcal{F} \) \\
        We should show that the event that the first player wins can be represented by a countable union of sets in \( \mathcal{F} \). The player goes first wins could happen in the following sequence: 1, 001, 00001... Therefore, F can be expressed by the union of these events. By hypothesis, these events are included in \( \mathcal{F} \). Therefore, \( F \in \mathcal{F} \).
        \item find \( P(F) \) \\
        According to the discussion above, we have 
        $$ P(F)=P(1)+P(001)+P(00001)+\dots=\frac{1}{2}+\frac{1}{8}+\dots=\sum_{k=1}^{\infty}\frac{1}{2^{2k-1}}=\frac{2}{3} $$
    \end{enumerate}
    That's the final answer.

\end{document}
